% add my chapters
% 
% It is strongly recommended that you read documentations located at
%   http://code.google.com/p/xjtuthesis/wiki/Landing?tm=6
% in advance of your compilation if you have not read them before.
%
% This work may be distributed and/or modified under the
% conditions of the LaTeX Project Public License, either version 1.3
% of this license or (at your option) any later version.
% The latest version of this license is in
%   http://www.latex-project.org/lppl.txt
% and version 1.3 or later is part of all distributions of LaTeX
% version 2005/12/01 or later.
%
% This work has the LPPL maintenance status `maintained'.
% 
% The Current Maintainer of this work is Weisi Dai.
%

%\chapter{溶菌酶的提取、分离纯化实验和条件优化方法}
\chapter{Lysozyme Extraction, Isolation and Purification Experiments and Condition Optimization Methods}

\section{Experimental Setup}
The following procedures are based on \cite{Yijun2020,Li-li2017,Yu-tong2006}.

\subsection{Experimental Reagents and Equipments}

Material: Fresh eggs, wall-warming micrococcus

Reagents: \ce{NaCl, NaH2PO4, NaOH, (NH4)2SO4}, concentrated hydrochloric acid

Comas Brilliant Blue G250, Comas Brilliant Blue R250, 0.45 μm filter membrane, CM Sepharose Fast Flow (GE), Sephacryl-S200 (GE), constant flow pump, UV monitor, UV-Vis spectrophotometer, fraction collector, chromatography data acquisition and processing system, glass chromatography column, centrifuge, acidometer.

In addition, we used different ratios of \ce{NaCl} to phosphate to configure Buffer for different pH values.

\hypertarget{pre-treatment-of-egg-white-samples}{%
\subsection{Pre-treatment of Egg White Samples}\label{pre-treatment-of-egg-white-samples}}

We start with an intact egg, wash the shell, and dry the outside of the shell. Afterwards, we gently crack the shell and pour the egg whites into a small beaker. With the yolk intact, we strain the egg whites through 2 layers of gauze to remove any umbilical chunks and broken shells, collecting the liquid. The egg whites are then diluted to 1.5 times their original volume with Buffer1 and stirred with a glass rod before being filtered through 6 layers of gauze. Finally, the supernatant was filtered through a 0.45 μm membrane and the filtrate was obtained.

\hypertarget{purification-process}{%
\subsection{Purification Process}\label{purification-process}}

We attempt to purify the lysozyme using cation exchange and then tested the purification by SDS-PAGE. After finding a suitable elution peak, we can use salt chromatography with resuspension to precipitate our desired enzyme from the sample. After purification by molecular sieve chromatography, we again check the results of the sieve purification by SDS-PAGE.

\hypertarget{concentration-measurement}{%
\subsection{Concentration Measurement}\label{concentration-measurement}}

We choose to use the Comas Brilliant Blue G250 staining method to determine the protein concentration of each sample.

\hypertarget{antibacterial-test}{%
\subsection{Antibacterial Test}\label{antibacterial-test}}

We first select the test species (Bacillus coccoidus, Bacillus cereus, E. coli), scribed on solid LB plate, incubated at 35 ℃ for 16h or so. After that, single colonies can be inoculated into liquid LB medium. After incubation under suitable conditions, we obtain a series of post-activation fluids. Finally, we directly add the same concentration of the sample obtained in previous experiments. The inhibition activity of lysozyme is determined by measuring the diameter of its inhibition loop.

%\section{Separation and Purification}

