\documentclass[a4paper,10pt]{article}

\usepackage[backend=bibtex,style=authoryear,natbib=true]{biblatex} % Use the bibtex backend with the authoryear citation style (which resembles APA)

\addbibresource{./Integrateexperimental.bib} % The filename of the bibliography

\usepackage[autostyle=true]{csquotes} % Required to generate language-dependent quotes in the bibliography

\begin{document}
    \section{Discussion on our application}
    Given the ability to lyse the cell wall of bacterial,
    lysozyme has been attracting immense attention as a kind of 
    environment-friendly or organs-friendly antimicrobial.
    As we have discussed before, it has nowadays more and more applications in the food industry and clinical procedure. In this part,  we will discuss 
    it's applications within our preparations and capabilities.
    \subsection{Applications in food industry}
    Our first focus is on the food industry. Since the 1800s when Napoléon launched his strategy to conquer Europe, the storage or preservation of food has become a major question in the food industry. Sealing food in cans, High-temperature treatment, Pasteurization, numerous methods had risen. And in the last, people introduced chemical additives to the food industry. Efficient as it is, nowadays people are having less tolerance in that chemical industry. Due to the conception of eating healthy, they prefer so-called "Non-additive food" to chemical treated food. But the lack of bacterial inhibition will easily make it a perfect bacterial petri dish. The chemical hazard vanishes, but the microbial hazard just arises. We need another method!
    
    So we cast our sight to the biological method to preserve food. In our case, we are planning to introduce lysozyme to food packing and food additives. As for the food packing, we are going to distribute the lysozyme agent, in gluten\citep{Conte2006} or onto a chitosan powder, on the food packages, mostly LDPE, these methods have been taken into practice\citep{Borzooeian2017}. And its function to extend to the shelf life of foods had been proved.\citep{Lian2012, Alhazmi2014} our points of view, this kind of application suits our capability very well, and we have put it into our first consideration.

    Another important application in the food industry is the food additive. We can add some lysozyme into specific easy-deteriorating foods,for example wurst,can-foods, and diary. The addition of lysozyme will significantly extend the preserve half-life of food, these lysozymes are presented into chitosan particals\citep{Wu2017}, this will not only protect the original flavour of the food but also enhance its ability to inhibit bacterial emerging.
    \subsection{Clinical applications}
    The lysozyme can also play an impressive role in the clinical procedure. Like the food industry, medical is also a battel against bacterial. In 1676, Anton van Leeuwenhoek observed bacteria and other microorganisms, using a single-lens microscope of his design.
    In 1796, Edward Jenner developed a method using cowpox to successfully immunize a child against smallpox. The same principles are used for developing vaccines today.
    Following on from this, in 1857 Louis Pasteur also designed vaccines against several diseases such as anthrax, fowl cholera and rabies as well as pasteurization for food preservation.
    In 1867 Joseph Lister is considered to be the father of antiseptic surgery. By sterilizing the instruments with diluted carbolic acid and using it to clean wounds, post-operative infections were reduced, making surgery safer for patients.
    In 1929 Alexander Fleming developed the most commonly used antibiotic substance both at the time and now: penicillin.\citep{Brock2003}.The emerge of antibiotic medicine start a new era for human, we can sometimes beat the infection of microbes.

    But there still exists a fatal problem: ALLERGY. Some antibiotics will lead to an acute allergic phenomenon, which is sometimes fatal. So we come up with this idea to introduce lysozyme to health-care products. We want to introduce it in for instance dentifrices, mouth-rinses, moisturizing gels, chewing gums or such sterilization products.\citep{Tenovuo2002}
    We try to develop a kind of lysozyme covered bandage in which the lysozyme exist in gel, or in other advanced status, such as carbon nanotubes.
    These are our prospects of the clinical application of lysozyme
    \printbibliography[title={Reference}] 
\end{document}