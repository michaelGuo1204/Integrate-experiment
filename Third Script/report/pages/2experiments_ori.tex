% add my chapters
% 
% It is strongly recommended that you read documentations located at
%   http://code.google.com/p/xjtuthesis/wiki/Landing?tm=6
% in advance of your compilation if you have not read them before.
%
% This work may be distributed and/or modified under the
% conditions of the LaTeX Project Public License, either version 1.3
% of this license or (at your option) any later version.
% The latest version of this license is in
%   http://www.latex-project.org/lppl.txt
% and version 1.3 or later is part of all distributions of LaTeX
% version 2005/12/01 or later.
%
% This work has the LPPL maintenance status `maintained'.
% 
% The Current Maintainer of this work is Weisi Dai.
%

%\chapter{溶菌酶的提取、分离纯化实验和条件优化方法}
\chapter{Lysozyme Extraction, Isolation and Purification Experiments and Condition Optimization Methods}

\section{Experimental Principles}
\subsection{Ion exchange chromatography}

Ion exchange chromatography is a widely applied experimental technique to separate substances. We firstly set up a glass column and add the original solution with a solute of interest
into it. Fixing a kind of high molecular weight (HMW) substances called ion exchanger on
the inner wall of the glass column, their exchangeable groups (or ions) go into the solution, and
some components of the solutes are absorbed onto the ion exchanger. (Here we apply cation
exchange chromatography, where the exchangeable group is cation) This is call exchange re­
action. The reaction is reversible and follows Le Chatelier’s principle. When an exchangeable
group disassociates from the column wall and its concentration in the solution rises, the solu­tion flows away and a new solution without this group or ion comes, pushing the reaction going
towards the positive direction. Meanwhile, the solute we are interested in is all absorbed onto
the column wall. Then we use another kind of solution to wash off (elute) this solute and get a pure solution of it.

The experiment can be divided into four phases:
\begin{enumerate}
\item balance: adding a solution to achieve the balance between ion exchanger and its exchangeable groups;
\item absorption: the exchange reaction;
\item elution: using buffers with different concentration to wash off substances in the order of low to high  affinity;
\item regeneration: use the original buffer (the solution to balance columns) to recover the columns.
\end{enumerate}

The effectiveness of separation is determined by the affinity of exchangeable groups and the solute. Furthermore, this affinity may change as the physicochemical properties (like pH,
salt concentration) change. So a careful choice of separation condition is important. Here
we add NaCl into the buffer to lower the solubility and promote the absorption. Thus, other
components will be absorbed in a farther position, making it easy to separate them.


\subsection{SDS-PAGE}

SDS-­PAGE (sodium dodecyl sulfate-polyacrylamide gel electrophoresis) is a commonly
used method to separate proteins with different molecular weights. Proteins are mixed with
SDS which denaturate them into rodlike macromolecules. SDS is a highly negative-­charged
micromolecule. It binds to proteins according to molecular weight (MW) of the protein and
forms a negative­charged complex. We put protein into an electrostatic field, and their only
difference is their size (MW).

Proteins will firstly be concentrated and pass through holes on PAGE, which is a crosslink­ing polymer. The bigger a protein is, the harder it moves forward and passes the structure, and
the later it reaches the anode. Thus we can separate different sized proteins. Using the Comas
Brilliant Blue R250 staining method, we can see the distribution of proteins. If the band of
our target protein is narrow and light without other bands near it, our product is pure, and its
concentration is high.

\subsection{Salting Out}

This method is often used to separate and purify biomarcomolecules. Adding high concentration inorganic salt (like \ce{(NH4)2SO4}, \ce{NaCl}) into the solution of HMW substances will
decrease their solubility and precipitate them. Those salts dissolve very well, deconstructing
the water layer on the surface of HMW substances and neutralizing their charge. Thus, they are
separated from other solutes.

\subsection{Gel Filtration Chromatography}

This is another usual separation method, which is extensively used because of its mild operational conditions. It also uses a reticular polymer (usually glucan or agarose), but the principle is different from SDS-PAGE. The bigger a protein is, the harder it is trapped in the hole, and the faster it reaches theother end of the column. Other micromolecules or smaller proteins stay in the hole and are separated from target proteins. 

The experiment can be divided into four steps:
\begin{enumerate}
\item swelling: put the dry gel into water or eluent;
\item filling: add gel to the column evenly (very important);
\item balancing, and loading sample;
\item finding the smaple peak, and collecting;
\item recycling the gel.
\end{enumerate}

The effectiveness of separation is mainly decided by molecular weight and affected by other conditions. We also need to carefully choose the type of gel, the length of the column, the buffer and so on. 

\section{Experimental Supplies}
See Table \ref{tab:glass},\ref{tab:instrument},\ref{tab:consumables},and \ref{tab:regents}.

\begin{table}[!h]
	\centering
	\caption{glass instrument}
    \begin{tabular}{lll}
    \toprule
	Name & Specifications & Quantity \\
	\midrule
	Beaker & 50    & 2 \\
	& 200   & 2 \\
	& 500   & 2 \\
	Glass rod &       & 2 \\
	Funnel &       & 1 \\
	Quartz cuvette &       & 4 \\
	Gel column &       & 1 \\
	26 mm glass chromatography column &       & 1 \\
	&       &  \\
	Test tube &       & 6 \\
	Iron frame platform &       & 1 \\
	Pipette & 1000  & 1 \\
	& 200   & 1 \\
	& 20    & 1 \\
	\bottomrule
	\end{tabular}%
\label{tab:glass}%
\end{table}%

\begin{table}[!h]
  \centering
\caption{instruments}
\begin{tabular}{ll}
	\toprule
	Name & Quantity \\
	\midrule
	Ultraviolet visible spectrophotometer & 1 \\
	Protein purification system & 1 \\
	Including pump, collector and detector &  \\
	EPS 601 DC regulated power supply & 1 \\
	Vertical plate electrophoresis tank & 1 \\
	Freeze vacuum dryer & 1 \\
	pH meter & 1 \\
	\bottomrule
\end{tabular}%
\label{tab:instrument}%
\end{table}%

\begin{table}[!h]
  \centering
\caption{consumables}
\begin{tabular}{lll}
	\toprule
	 Name & Specifications & Quantity \\
	 \midrule
	hen eggs &       & 40 \\
	Gauze & package & 1 \\
	0.45 μm filter membrane & package & 1 \\
	1.5 mL centrifuge tube & package & 1 \\
		\bottomrule
\end{tabular}%
\label{tab:consumables}%
\end{table}%


\begin{table}[!h]
	  \centering
	\caption{reagents}
	\begin{tabular}{lll}
		\toprule
	Name & Specifications & Quantity \\
	\midrule
	\tabincell{c}{50\% CM Sepharose Fast Flow\\ (ion exchange suspension)} & ml    & 50 \\
	\ce{(NH4)2SO4} & g     & 100 \\
	50\% Sephacryl-S200 medium &       &  \\
	SDS   & g     & 50 \\
	Ammonium persulfate (AP) & g     & 5 \\
	TEMED & ml    & 2 \\
	Tris  & g     & 20 \\
	Glycine   & g     & 94 \\
	HCl   & ml    & 200 \\
	n-butanol & ml    & 5 \\
	0.5\% bromophenol blue & ml    & 10 \\
	50\% glycerol & ml    & 100 \\
	Coomassie Brilliant Blue R250 & ml    & 20 \\
	5\%β- mercaptoethanol & ml    & 10 \\
	Methanol & L     & 1 \\
	Glacial acetic acid & ml    & 200 \\
	NaOH  & g     & 20 \\
	\tabincell{c}{0.5mg/ml of standard protein solution\\ (bovine serum albumin solution)} & ml    & 5 \\
	Coomassie Brilliant Blue G250 & ml    & 100 \\
	\ce{NaCl}  & g     & 100 \\
	\ce{NaH2PO4} & g     & 500 \\
	\ce{Na2HPO4} & g     & 200 \\
	\textit{Bacillius. subtilis} &      &  \\
		\bottomrule
\end{tabular}%
\label{tab:regents}%
\end{table}%



\section{Experimental Setup}
The following procedures are based on \cite{Yijun2020,Li-li2017} and \cite{Yu-tong2006}.

\subsection{Experimental Reagents and Equipments}

Material: Fresh eggs, \textit{Bacillius. subtilis}

Salt solutions: \ce{NaCl, NaH2PO4, Na2HPO4, NaOH, (NH4)2SO4}, concentrated hydrochloric acid

Reagents: Comas Brilliant Blue G250, Comas Brilliant Blue R250, 0.45 μm filter
membrane, CM Sepharose Fast Flow (GE), Sephacryl-S200 (GE), constant
flow pump, UV monitor, UV-Vis spectrophotometer, fraction collector,
chromatography data acquisition and processing system, glass
chromatography column, centrifuge, acidometer.


In addition, we add different ratios of \ce{NaCl} to the phosphate to prepare for Buffer.

\subsection{Buffer}

\begin{itemize}
	\item Buffer1:pH=4.5,0.1 mol/L phosphate,50 mmol/L NaCl 
	\item Buffer2:pH=4.5,0.1 mol/L phosphate,200 mmol/L NaCl 
	\item Buffer3:pH=4.5,0.1 mol/L phosphate,500 mmol/L NaCl 
	\item Buffer4:pH=4.5,0.1 mol/L phosphate,1 mol/L NaCl 
	\item Buffer5:pH=7.0,0.1 mol/L phosphate,50 mmol/L NaCl
\end{itemize}

\subsection{Pre-treatment of Egg White Samples}

We start with an intact egg, wash the shell, and dry the outside of the
shell. Afterwards, we gently crack the shell and pour the egg whites
into a small beaker. With the yolk intact, we strain the egg whites
through 2 layers of gauze to remove any umbilical chunks and broken
shells, collecting the liquid. The egg whites are then diluted to 1.5
times their original volume with Buffer1 and stirred with a glass rod
before being filtered through 6 layers of gauze. Finally, the
supernatant was filtered through a 0.45 μm membrane and the filtrate was
obtained.

\hypertarget{purification-process}{%
	\subsection{Purification Process}\label{purification-process}}

We attempt to purify the lysozyme using cation exchange and then
tested the purification by SDS-PAGE. After finding a suitable elution
peak, we can use salt chromatography with resuspension to precipitate
our desired enzyme from the sample. After purification by molecular
sieve chromatography, we again check the results of the sieve
purification by SDS-PAGE.

\subsubsection{Ion exchange chromatography}

We take about 15 mL of 50\% CM Sepharose Fast Flow ion exchange
suspension and load it onto a 26 mm diameter glass column to check the
seal.

After that, we use Buffer1 to balance the ion exchange column, adjusted
the constant flow rate of the constant flow pump to 1 mL/min, balanced
for about 30 min (about 4-5 column volumes), and finally stabilized the
UV absorption curve at the baseline.

We then load Sample1 at a flow rate of 0.5 mL/min and rinsed with
Buffer1 at a flow rate of 1 mL/min. Record the OD280-time curve and
observe the elution peaks as they occur. Stop the flush when OD280
drops close to baseline and remains unchanged.

We then elute with Buffer2 at a flow rate of 2 mL/min and open the
fraction collector for fraction collection (1 mL/tube).

When the OD280 returns to baseline, we switch to Buffer3 for elution at
2 mL/min flow rate.

After elution, we rinse 5 column volumes of regeneration medium with
Buffer4 at a flow rate of 2 mL/min and then use Buffer1 at a flow rate
of 1 mL/min to wash away the residual high salt solution in the column
to complete column regeneration.

Finally, the elution peaks at different Buffer are obtained.

\subsubsection{Using SDS-PAGE to detect purification effects}

Denaturing polyacrylamide gels (15\% separated gel, 4\% concentrated
gel) are prepared. We then electrophorese a small amount of each
fraction obtained from ion exchange chromatography separately and
display the electrophoresis results using the Comas Brilliant Blue
R250 staining method. We determine the lysozyme bands based on
molecular weight and the purity of lysozyme in the eluted fraction based
on the electrophoresis results, and selected the best elution peak as
sample2.

\subsubsection{Salting and resuspension}

We dissolve 3.5 mg of \ce{(NH4)2SO4} powder in 10 mL of Sample2 in several
small amounts and stirred with a glass rod while adding the powder to
prevent local salt concentration from excessive hetero protein salting.
The suspension was divided into 1.5 mL centrifuge tubes and centrifuged
at 12,000 rpm for 10 min, the supernatant from each tube was discarded.
Finally, resuspend with 1 mL of Buffer5 and combine all the
precipitates.

\subsubsection{Molecular Sieve Chromatography}

We take an appropriate amount of 50\% Sephacryl-S200 medium and added
Buffer5 at a flow rate of 0.8 mL/min to equilibrate the medium on a
glass chromatography column. The sample was then loaded and eluted with
Buffer5, while the fraction collector was opened for peak collection (1
mL/tube). Record OD280-time curve. When the OD280 returns to baseline
and elutes beyond one column volume, stop the elution. Finally determine
the fraction absorption peak corresponding to the lysozyme.

\subsubsection{Detection of molecular sieve effects by SDS-PAGE}

We first observe the results of molecular sieve purification by
SDS-polyacrylamide gel electrophoresis analysis, after which we select
a single fraction of the electrophoretic band and mixed it well as
Sample3 (measured volume).

\hypertarget{concentration-measurement}{%
	\subsection{Concentration Measurement}\label{concentration-measurement}}

We choose to use the Comas Brilliant Blue G250 staining method to
determine the protein concentration of each sample.

\hypertarget{antibacterial-test}{%
	\subsection{Antibacterial Test}\label{antibacterial-test}}

\subsubsection{Preparation of standard solution for \textit{Bacillus. subtilis}}

Culture \textit{Bacillus. subtilis} it in a sterilized mortar, decant and dilute the substrate suspension to the desired absorbance value.

\subsubsection{Determination of lysozyme activity}

Add 2.5 mL of substrate suspension and 0.5 mL of standard enzyme
solution to a 1 cm cell at 25℃, mix well and count the blanks, measure
the absorbance every 15 s for a total of 6 times. pH=6.4).

\begin{longtable}[]{@{}llll@{}}
	\toprule
	Sample & Sample1 & Sample2 & Sample3\tabularnewline
	\midrule
	\endhead
	Micrococcus lysozyme & 800 & 800 & 800\tabularnewline
	Buffer5 & 195 & 195 & 185\tabularnewline
	Sample volume & 5 & 5 & 5\tabularnewline
	\bottomrule
\end{longtable}

\subsubsection{Definition of enzyme activity unit.}

At room temperature 25°C, pH 6.2 and a wavelength of 450 nm, a
decrease of 0.001 in absorption caused by each min is one enzyme
activity unit, then the activity unit U per mg of lysozyme is
\begin{gather*}
	Enzyme Activity(\mathrm{U\cdot mL^{-1}})=\dfrac{OD_{420}(60)-OD_{420}(0)}{v_0}\times 10^{6}
\end{gather*}


\section{Our Innovation}
The extraction and the purification of lysozyme have been widely and deeply researched, there have existed many useful protocols and measurements concerning lysozyme. So to demonstrate our innovation, we cast our sight to the application of lysozyme. 

As for the food industry, we are planning a Lysozyme-contained polygenic Sphere. This kind of product will present very attracting characteristics such as low health hazard, controlled lysozyme release and the most important: high stability. All these properties will make our product help the food industry improve their food’s half-shelf life without introducing hazardous chemicals. We have designed a polygon-sphere with unregular-shaped holes on it. The polygon-made skeleton will provide a stable foundation to the product while the unregular hole will provide adequate specific surface area to the efficient working of lysozyme.

\begin{figure}[!h]
	\centering
	\includegraphics[width=0.7\linewidth]{figures/chp2_innovation}
	\caption{a skecth of our polygon-sphere}
	\label{fig:chp2innovation}
\end{figure}


In the articles we have referred to, they always introduce a soft or gel-like material due to the excellent controlled-release properties. We decide to use the hard polygenic material while equipping holes on it, we think that this will equip our product higher stability and get rid of the problems by the dissolve of the gel material. 

Apart from the food industry, we also come up with ideas applying our products to medicine. We can inlay our microsphere to the bandage, and make our bandage present anti-microbe properties, which will minimize the frequency of re-applying the bandage.

The applications of lysozyme are being more and more popular recently, and we hope our methods will provide a new idea of applying lysozyme in our daily lives. 
