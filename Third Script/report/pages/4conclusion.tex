% multiple1902 <multiple1902@gmail.com>
% conclusion.tex
% Copyright 2011~2012, multiple1902 (Weisi Dai)
% https://code.google.com/p/xjtuthesis/
% 
% It is strongly recommended that you read documentations located at
%   http://code.google.com/p/xjtuthesis/wiki/Landing?tm=6
% in advance of your compilation if you have not read them before.
%
% This work may be distributed and/or modified under the
% conditions of the LaTeX Project Public License, either version 1.3
% of this license or (at your option) any later version.
% The latest version of this license is in
%   http://www.latex-project.org/lppl.txt
% and version 1.3 or later is part of all distributions of LaTeX
% version 2005/12/01 or later.
%
% This work has the LPPL maintenance status `maintained'.
% 
% The Current Maintainer of this work is Weisi Dai.
%

%\chapter{结论与展望}
\chapter{Conclusion and Prospects}
%
%    \xjtuthesis 是一个开源项目,旨在提供符合西安交通大学有关部门要求的学位论文\LaTeX 模板。
%
%    您当前看到的文件是 \xjtuthesis{} \metaversion 的示例排版文档。
%
%    \xjtuthesis 项目目前托管在Google Code: \verb|http://xjtuthesis.googlecode.com/|
%    
%    \xjtuthesis 采用Mercurial管理源代码。访问项目的网站,了解更多信息。
%
%    \section{使用\xjtuthesis}
%
%        如果你是本科生,请和系里联系以确定可以使用\xjtuthesis 完成论文。
%
%        研究生请和西安交大研究生院学位办(周主任)联系:
%        
%        \begin{description}
%            \item[电话] 82668899
%            \item[办公地址] 教学主楼1311室
%            \item[电子邮件] \url{xwb@mail.xjtu.edu.cn} (从来不回)
%        \end{description}


\section{Experimental Success}
The major success we achieved is to separate lysozyme with bacteriolysis activity from egg white through cation exchange chromatography. 
\begin{itemize}
	\item We chose a kind of cheap filler and conducted very simple procedures to collect enough expected product. \item We found the experiment is easily repeatable, no matter through a smaller or bigger column.
	\item Our experimental procedures helped us achieve a recovery rate surprisingly close to 100\% and high specific activity ($13.4\mathrm{U\cdot mg^{-1}}$).
	\item We got high concentration lysozyme solution by condensing the eluate with polyethylene glycol 20000, which enabled us to test the activity easily.
	\item We tried to gain insight into our obtained data, fitting with many models.
	\item We also helped many other groups to successfully conducted G100 Gel Filtrition, SDS-PAGE and staining experiments.
\end{itemize} 

\section{Experimental Prospects}

Though it is great to get the product we want, there is still a lot of room for imporvement.
\begin{itemize}
	\item
	Try to adjust conditions and extract lysozyme with G100. By changing the column length or other conditions, we may succeed. But actually the content of lysozyme is not big enough to guarantee this.
	\item
	Further purification may be conducted. We did not conduct salting out or recrystallization, neither did we measure the purity of lysozyme in the eluate.
	\item
	Explore the effect of pH with accurate control. This is a major exception and failure. Due to time limit, we did not redo this experiment.
	\item
	Collect more data for kinetics analysis (combined with theoretical
	explanation), or adjust the enzyme concentration. 
	\item
	More property analysis (like optimal temperature). This is also
	valuable for us to identify the purity and compare with the
	literature.
\end{itemize}

We think for scientists and manufacturers, improvement in separation procedures is not the most necessary. Most research is needed in property study, molecular modification and improvement, and other production-lifting methods.

\section{Our Innovation}
The extraction and the purification of lysozyme have been widely and deeply researched, there have existed many useful protocols and measurements concerning lysozyme. 
To demonstrate our innovation, we cast our sight to the application of lysozyme. 

As for the food industry, we are planning a ysozyme-contained Polygenic Sphere. This kind of product will present very attracting characteristics such as low health hazard, controlled lysozyme release and the most important: high stability. All these properties will make our product help the food industry improve their food’s half-shelf life without introducing hazardous chemicals. We have designed a polygon-sphere with unregular-shaped holes on it. The polygon-made skeleton will provide a stable foundation to the product while the unregular hole will provide adequate specific surface area to the efficient working of lysozyme.

\begin{figure}[!h]
	\centering
	\includegraphics[width=0.7\linewidth]{figures/chp2_innovation}
	\caption{a skecth of our polygon-sphere}
	\label{fig:chp2innovation}
\end{figure}


In the articles we have referred to, they always introduce a soft or gel-like material due to the excellent controlled-release properties. We decide to use the hard polygenic material while equipping holes on it, we think that this will equip our product higher stability and get rid of the problems by the dissolve of the gel material. 

Apart from the food industry, we also come up with ideas applying our products to medicine. We can inlay our microsphere to the bandage, and make our bandage present anti-microbe properties, which will minimize the frequency of re-applying the bandage.

The applications of lysozyme are being more and more popular recently, and we hope our methods will provide a new idea of applying lysozyme in our daily lives. 

