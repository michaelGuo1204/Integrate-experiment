% add my chapters
% 
% It is strongly recommended that you read documentations located at
%   http://code.google.com/p/xjtuthesis/wiki/Landing?tm=6
% in advance of your compilation if you have not read them before.
%
% This work may be distributed and/or modified under the
% conditions of the LaTeX Project Public License, either version 1.3
% of this license or (at your option) any later version.
% The latest version of this license is in
%   http://www.latex-project.org/lppl.txt
% and version 1.3 or later is part of all distributions of LaTeX
% version 2005/12/01 or later.
%
% This work has the LPPL maintenance status `maintained'.
% 
% The Current Maintainer of this work is Weisi Dai.
%

%\chapter{溶菌酶的提取、分离纯化实验和条件优化方法}
\chapter{Lysozyme Separation and Purification Experiments and Condition Optimization Methods}

\section{Principles of Separation Experiments}
\subsection{Ion exchange chromatography}

Ion exchange chromatography is a widely applied experimental technique to separate substances. We firstly set up a glass column and add the original solution with a solute of interest
into it. Fixing a kind of high molecular weight (HMW) substances called ion exchanger on
the inner wall of the glass column, their exchangeable groups (or ions) go into the solution, and
some components of the solutes are absorbed onto the ion exchanger. (Here we apply cation
exchange chromatography, where the exchangeable group is cation) This is call exchange re­
action. The reaction is reversible and follows Le Chatelier’s principle. When an exchangeable
group disassociates from the column wall and its concentration in the solution rises, the solu­tion flows away and a new solution without this group or ion comes, pushing the reaction going
towards the positive direction. Meanwhile, the solute we are interested in is all absorbed onto
the column wall. Then we use another kind of solution to wash off (elute) this solute and get a pure solution of it.

The experiment can be divided into four phases:
\begin{enumerate}
\item balance: adding a solution to achieve the balance between ion exchanger and its exchangeable groups;
\item absorption: the exchange reaction;
\item elution: using buffers with different concentration to wash off substances in the order of low to high  affinity;
\item regeneration: use the original buffer (the solution to balance columns) to recover the columns.
\end{enumerate}

The effectiveness of separation is determined by the affinity of exchangeable groups and the solute. Furthermore, this affinity may change as the physicochemical properties (like pH,
salt concentration) change. So a careful choice of separation condition is important. Here
we add NaCl into the buffer to lower the solubility and promote the absorption. Thus, other
components will be absorbed in a farther position, making it easy to separate them.


\subsection{SDS-PAGE}

SDS-­PAGE (sodium dodecyl sulfate-polyacrylamide gel electrophoresis) is a commonly
used method to separate proteins with different molecular weights. Proteins are mixed with
SDS which denaturate them into rodlike macromolecules. SDS is a highly negative-­charged
micromolecule. It binds to proteins according to molecular weight (MW) of the protein and
forms a negative­charged complex. We put protein into an electrostatic field, and their only
difference is their size (MW).

Proteins will firstly be concentrated and pass through holes on PAGE, which is a crosslink­ing polymer. The bigger a protein is, the harder it moves forward and passes the structure, and
the later it reaches the anode. Thus we can separate different sized proteins. Using the Comas
Brilliant Blue R250 staining method, we can see the distribution of proteins. If the band of
our target protein is narrow and light without other bands near it, our product is pure, and its
concentration is high.

%\subsection{Salting Out}
%
%This method is often used to separate and purify biomarcomolecules. Adding high concentration inorganic salt (like \ce{(NH4)2SO4}, \ce{NaCl}) into the solution of HMW substances will
%decrease their solubility and precipitate them. Those salts dissolve very well, deconstructing
%the water layer on the surface of HMW substances and neutralizing their charge. Thus, they are
%separated from other solutes.

\subsection{Gel Filtration Chromatography}

This is another usual separation method, which is extensively used because of its mild operational conditions. It also uses a reticular polymer (usually glucan or agarose), but the principle is different from SDS-PAGE. The bigger a protein is, the harder it is trapped in the hole, and the faster it reaches theother end of the column. Other micromolecules or smaller proteins stay in the hole and are separated from target proteins. 

The experiment can be divided into four steps:
\begin{enumerate}
\item swelling: put the dry gel into water or eluent;
\item filling: add gel to the column evenly (very important);
\item balancing, and loading sample;
\item finding the sample peak, and collecting;
\item recycling the gel.
\end{enumerate}

The effectiveness of separation is mainly decided by molecular weight and affected by other conditions. We also need to carefully choose the type of gel, the length of the column, the buffer and so on. 

\section{Experimental Supplies}
See Table \ref{tab:glass},\ref{tab:instrument},\ref{tab:consumables},and \ref{tab:regents}.

\begin{table}[!h]
	\centering
	\caption{Glass instrument}
    \begin{tabular}{lll}
    	\toprule
    	Name  & Specifications & Quantity \\
    	\midrule
    	Beaker & 50    & 2 \\
    	& 200   & 2 \\
    	& 500   & 2 \\
    	Glass rod &       & 2 \\
    	funnel &       & 1 \\
    	quartz cuvette &       & 4 \\
    	&       & 1 \\
    	&       & 1 \\
    	&       &  \\
    	Test tube &       & 6 \\
    	Iron support &       & 1 \\
    	\bottomrule
    \end{tabular}%
	\label{tab:glass}%
\end{table}%

\begin{table}[!h]
  \centering
\caption{Other instruments}
\begin{tabular}{ll}
	\toprule
	Name & Quantity \\
	\midrule
	Ultraviolet visible spectrophotometer & 1 \\
	\tabincell{c}{Protein purification system \\ \small Including pump, collector and detector} & 1 \\
	EPS 601 DC regulated power supply & 1 \\
	Vertical plate electrophoresis tank & 1 \\
	pH meter & 1 \\
	\bottomrule 
\end{tabular}%
\label{tab:instrument}%
\end{table}%

\begin{table}[!h]
  \centering
\caption{Consumables}
\begin{tabular}{lll}
	\toprule
	 Name & Specifications & Quantity \\
	 \midrule
	Hen eggs &       & 6 \\
	Gauze & package & 1 \\
	0.45 μm filter membrane & package & 1 \\
	1.5 mL centrifuge tube & package & 1 \\
	Pipette tips(20μL,200μL,1mL) & package & 3 \\
		\bottomrule
\end{tabular}%
\label{tab:consumables}%
\end{table}%


\begin{table}[!h]
	  \centering
	\caption{Reagents}
	\begin{tabular}{lll}
		\toprule
	Name & Specifications & Quantity \\
	\midrule
	    \ce{(NH4)2SO4} & g     & 100 \\
	SDS   & g     & 50 \\
	Ammonium persulfate (AP) & g     & 5 \\
	TEMED & ml    & 2 \\
	Tris  & g     & 20 \\
	Gly   & g     & 94 \\
	HCl   & ml    & 200 \\
	n - butanol & ml    & 5 \\
	0.5\% bromophenol blue & ml    & 10 \\
	50\% glycerol & ml    & 100 \\
	Coomassie Brilliant Blue R250 & ml    & 20 \\
	5\%β- mercaptoethanol & ml    & 10 \\
	Methanol & L     & 1 \\
	Glacial acetic acid & ml    & 200 \\
	NaOH  & g     & 20 \\
	NaCl  & g     & 150 \\
	\ce{NaH2PO4} & g     & 500 \\
	\ce{Na2HPO4} & g     & 200 \\
	\textit{Bacillius. subtilis} suspension & μL    & 10 \\
	tryptone & g     & 10 \\
	yeast extract & g     & 5 \\
	Sephadex gel G100 & g     & 2 \\
	Cation exchange packing 732 & g     & 500 \\
	polyethyleneglycol 20000 & g     & 100 \\
		\bottomrule
\end{tabular}%
\label{tab:regents}%
\end{table}%

\section{Experimental Setup}
The following procedures are based on \cite{Liu2020,Li-li2017} and \cite{Yu-tong2006}.

\hypertarget{header-n5}{%
\subsection{Experimental Reagents and Equipments}\label{header-n5}}

Material: Fresh eggs, \emph{Bacillius. subtilis}

Salt solutions: \ce{NaCl, NaH2PO4, Na2HPO4, NaOH, K2HPO4}, concentrated
hydrochloric acid

Reagents: Comas Brilliant Blue G250, Comas Brilliant Blue R250, 0.45 μm
filter membrane, 
constant flow pump, UV monitor, UV-­Vis spectrophotometer, fraction
collector, chromatography data acquisition and processing system, glass chromatography column, centrifuge, acidometer.

\hypertarget{header-n9}{%
\subsection{Buffer}\label{header-n9}}

Buffer1:pH=4.5, 0.1 mol/L phosphate, 50 mmol/L NaCl

Buffer2:pH=4.5, 0.1 mol/L phosphate, 1mol/L NaCl

Buffer3:pH=9.0, 0.1 mol/L phosphate

Buffer4:pH=9.0, 0.1 mol/L phosphate, 1mol/L NaCl

Resuspension PBS (0.02 mol/L): pH=4.5, NaCl 8g/L, KCl 0.2g/L, \ce{Na2HPO4} 1.42g/L, \ce{KH2PO4} 0.27 g/L

\hypertarget{header-n13}{%
\subsection{Pretreatment of Egg White Samples}\label{header-n13}}

We start with an intact egg, wash the shell, and dry the outside of the
shell. Afterwards, we gently crack the shell and pour the egg whites
into a small beaker. With the yolk intact, we strain the egg whites
through 2 layers of gauze to remove any umbilical chunks and broken
shells, collecting the liquid. The egg whites are then diluted to 1.5
times their original volume with Buffer1 and stirred with a glass rod
before being filtered through 6 layers of gauze. Finally, the
supernatant was filtered through a 0.45 μm membrane and the filtrate was
obtained.

\hypertarget{header-n15}{%
\subsection{Purification Process}\label{header-n15}}

\hypertarget{header-n16}{%
\subsubsection{Separation through Glucan G100}\label{header-n16}}

Sephadex G100 was soaked in normal saline at 100 ℃, and the supernatant
was poured out after soaking.

Then we took buffer 1 and loaded it into a 10 mm diameter glass
chromatographic column. The chromatographic column was connected with
constant flow pump, ultraviolet monitor and fraction collector. Turn on
the constant flow pump and check the tightness.

We poured the treated Sephadex G100 into the column at one time, used buffer 1
to balance the column, adjusted the constant flow pump to
control the flow rate to 5 ml/min, and balanced the column until the
UV absorption curve was stable at the baseline.

We loaded sample at the flow rate of 0.5 ml/min, and the loading volume was 1 ml. Then flush with buffer 1 at a flow rate of 0.5 ml/min. The $\mathrm{OD_{280}}$ time curve was recorded, the elution peaks were observed,
and the eluates of each fraction were collected. When $\mathrm{OD_{280}}$ dropped
close to the baseline and remained unchanged, the flushing was stopped.

%Finally, the elution peak was obtained as follows.

%\includegraphics{/C:/Users/15993/AppData/Local/Temp/ksohtml10220/wps1.jpg}

\hypertarget{header-n23}{%
\subsubsection{Detection of Gel Filtration Effects by SDS-PAGE \citep{SDSfor}}}

Prepare denatured polyacrylamide gel (15\% separating glue and 10\%
concentrated glue). The fractions obtained by ion exchange
chromatography were electrophoresed and the electrophoretic results were
displayed by Coomassie brilliant blue R250 staining. The band of lysozyme was determined by molecular weight and the purity  was represented by electrophoresis results. 
%The best elution peak was selected as sample 2

%The results are as follows, the results are not ideal

%\includegraphics{/C:/Users/15993/AppData/Local/Temp/ksohtml10220/wps2.jpg}

\subsubsection{732 Cation Resin Separation \citep{Hu2015,Lin2002,732}}

We took 100 g cation exchange resin (type 732%, type D061
) and put it in a
beaker. Wash it repeatedly with distilled water until there is no obvious impurity and filter it. Then we added 1mol/L HCl solution 2-4
times the volume of resin and stir it for 1 hours and pour out the
supernatant. Filter it dry. After that, we added 1 mol/L NaOH solution 2-4 times the volume of the resin, stirred it for 1 hours, poured out the supernatant, washed it repeatedly with distilled water.

Buffer 3 was loaded into a 10 mm diameter glass chromatographic column.
The chromatographic column was connected with constant flow pump,
ultraviolet monitor and fraction collector. Turn on the constant flow
pump and check the tightness.

The treated 732 cation exchange resin was poured into the column, and buffer 3 was used to balance the column. The constant flow pump was
adjusted to control the flow rate to 3 ml / min, and the column was
balanced until the UV absorption curve was stable at the baseline.

Our sample was loaded on the smaller column at a flow rate of 2 ml/min, with a sample loading
volume of about 1 ml. Then flush with buffer 3 at a flow rate of 3 ml/min
until the UV absorption curve is stable at the baseline.

Buffer 4 was used for the second flushing operation at the flow rate of
5 ml/min. when the elution peak began to appear, the collection began
until the UV absorption curve stabilized at the baseline.

\subsubsection{Sample Concentration \citep{condense}}

The eluate is so diluted that we concentrate the samples by putting them into water absorbent Polyethylene glycol 20000 overnight to better measure their properties.

The dialysis bag was treated in boiling water bath for 20 minutes, and
was taken out for standby.

After that, we added the protein solution into the dialysis bag and embedded it in polyethylene glycol. After 10 hours, the dialysate was
collected for post-treatment.

\hypertarget{header-n36}{%
\subsubsection{Verification of Ion Exchange Chromatography by SDS-PAGE
}\label{header-n36}}

SDS- polyacrylamide gel electrophoresis was used to analyze the results
of ion exchange purification. 
%give the result as follows

%\includegraphics{/C:/Users/15993/AppData/Local/Temp/ksohtml10220/wps3.jpg}

\subsection{Concentration Measurement}

We use the Ultra-micro Protein Nucleic Acid Analyzer to directly determine the protein concentration of each sample.

\subsection{Antibacterial Test \citep{activity,Liu2020}}

We are doing two investigative experiments. With the standard enzyme solution, we explore the effect of pH on enzyme activity by varying pH of the resuspension buffer from 5.0 to 9.0 with the step of 1.0. Then we record the reaction curve (absorbance changing with time) of standard solution and concentrated eluates, from which we may find out the reaction kinetics.

\subsubsection{Preparation of standard solution for \textit{Bacillus. subtilis}}

\begin{itemize}
	\item Pipette 10µL of \textit{Bacillus. subtilis} culture into the broth. Culture it in a sterilized mortar overnight.
	\item Centrifuge at 4000rpm for 5min to remove the supernatant.
	\item The remaining bacteria were resuspended and diluted to the desired absorbance value (0.5$\sim$1.0) by 0.02M PBS.
\end{itemize}

\subsubsection{Determination of lysozyme activity}

\begin{itemize}%[label=\arabic*]
	\item Place the substrate suspension and standard enzyme solution (2g/L) at 25℃.
	\item Absorb 2.0 mL of substrate suspension and 0.4 mL of standard enzyme solution to a 1 cm cell, mix them quickly and record the absorbance once every 15 s. (use PBS buffer as the blank control)
\end{itemize}

%\begin{longtable}[]{@{}llll@{}}
%	\toprule
%	Sample & Sample1 & Sample2 & Sample3\tabularnewline
%	\midrule
%	\endhead
%	Micrococcus lysozyme & 800 & 800 & 800\tabularnewline
%	Buffer5 & 195 & 195 & 185\tabularnewline
%	Sample volume & 5 & 5 & 5\tabularnewline
%	\bottomrule
%\end{longtable}

\subsubsection{Definition of enzyme activity unit.}

At room temperature 25°C, designed pH and a wavelength of 600 nm, a decrease of 0.001 in absorption each min is one enzyme
activity unit, then the activity unit U per g/L of lysozyme is
\begin{gather*}
	EnzymeActivity(\mathrm{U\cdot mL^{-1}})=\dfrac{OD_{600}(t+60)-OD_{600}(t)}{Enzyme Solution Volume}\times 10^{3}\\
	SpecificActivity(\mathrm{U\cdot g^{-1} })=\dfrac{EnzymeActivity}{Enzyme Concentration(\mathrm{g\cdot mL^{-1}})}
\end{gather*}
