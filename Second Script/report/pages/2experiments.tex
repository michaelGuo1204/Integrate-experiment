% add my chapters
% 
% It is strongly recommended that you read documentations located at
%   http://code.google.com/p/xjtuthesis/wiki/Landing?tm=6
% in advance of your compilation if you have not read them before.
%
% This work may be distributed and/or modified under the
% conditions of the LaTeX Project Public License, either version 1.3
% of this license or (at your option) any later version.
% The latest version of this license is in
%   http://www.latex-project.org/lppl.txt
% and version 1.3 or later is part of all distributions of LaTeX
% version 2005/12/01 or later.
%
% This work has the LPPL maintenance status `maintained'.
% 
% The Current Maintainer of this work is Weisi Dai.
%

%\chapter{溶菌酶的提取、分离纯化实验和条件优化方法}
\chapter{Lysozyme Extraction, Isolation and Purification Experiments and Condition Optimization Methods}

\section{Experimental Principles}









\section{Experimental Supplies}











\section{Experimental Setup}
The following procedures are based on \cite{Yijun2020,Li-li2017} and {Yu-tong2006}.

\subsection{Experimental Reagents and Equipments}

Material: Fresh eggs, wall-warming micrococcus

Salt solutions: \ce{NaCl, NaH2PO4, NaOH, (NH4)2SO4}, concentrated hydrochloric acid

Reagents: Comas Brilliant Blue G250, Comas Brilliant Blue R250, 0.45 μm filter
membrane, CM Sepharose Fast Flow (GE), Sephacryl-S200 (GE), constant
flow pump, UV monitor, UV-Vis spectrophotometer, fraction collector,
chromatography data acquisition and processing system, glass
chromatography column, centrifuge, acidometer.


In addition, we used different ratios of \ce{NaCl} to phosphate to configure Buffer for different pH values.


\subsection{Pre-treatment of Egg White
		Samples}

We start with an intact egg, wash the shell, and dry the outside of the
shell. Afterwards, we gently crack the shell and pour the egg whites
into a small beaker. With the yolk intact, we strain the egg whites
through 2 layers of gauze to remove any umbilical chunks and broken
shells, collecting the liquid. The egg whites are then diluted to 1.5
times their original volume with Buffer1 and stirred with a glass rod
before being filtered through 6 layers of gauze. Finally, the
supernatant was filtered through a 0.45 μm membrane and the filtrate was
obtained.

\hypertarget{purification-process}{%
	\subsection{Purification Process}\label{purification-process}}

We attempted to purify the lysozyme using cation exchange and then
tested the purification by SDS-PAGE. After finding a suitable elution
peak, we can use salt chromatography with resuspension to precipitate
our desired enzyme from the sample. After purification by molecular
sieve chromatography, we again check the results of the sieve
purification by SDS-PAGE.

\subsubsection{Ion exchange chromatography}

We took about 15 mL of 50\% CM Sepharose Fast Flow ion exchange
suspension and loaded it onto a 26 mm diameter glass column to check the
seal.

After that, we used Buffer1 to balance the ion exchange column, adjusted
the constant flow rate of the constant flow pump to 1 mL/min, balanced
for about 30 min (about 4-5 column volumes), and finally stabilized the
UV absorption curve at the baseline.

We then loaded Sample1 at a flow rate of 0.5 mL/min and rinsed with
Buffer1 at a flow rate of 1 mL/min. Record the OD280-time curve and
observe the elution peaks as they occur, stopping the flush when OD280
drops close to baseline and remains unchanged.

We then elute with Buffer2 at a flow rate of 2 mL/min and open the
fraction collector for fraction collection (1 mL/tube).

When the OD280 returns to baseline, we switch to Buffer3 for elution at
2 mL/min flow rate.

After elution, we rinse 5 column volumes of regeneration medium with
Buffer4 at a flow rate of 2 mL/min and then use Buffer1 at a flow rate
of 1 mL/min to wash away the residual high salt solution in the column
to complete column regeneration.

Finally, the elution peaks at different Buffer were obtained.

\subsubsection{Using SDS-PAGE to detect purification effects}

Denaturing polyacrylamide gels (15\% separated gel, 4\% concentrated
gel) were prepared. We then electrophoresed a small amount of each
fraction obtained from ion exchange chromatography separately and
displayed the electrophoresis results using the Comas Brilliant Blue
R250 staining method. We determined the lysozyme bands based on
molecular weight and the purity of lysozyme in the eluted fraction based
on the electrophoresis results, and selected the best elution peak as
sample2.

\subsubsection{Salting and resuspension}

We dissolved 3.5 mg of (NH4)2SO4 powder in 10 mL of Sample2 in several
small amounts and stirred with a glass rod while adding the powder to
prevent local salt concentration from excessive hetero protein salting.
The suspension was divided into 1.5 mL centrifuge tubes and centrifuged
at 12,000 rpm for 10 min, the supernatant from each tube was discarded.
Finally, resuspend with 1 mL of Buffer5 and combine all the
precipitates.

\subsubsection{Molecular Sieve Chromatography}

We took an appropriate amount of 50\% Sephacryl-S200 medium and added
Buffer5 at a flow rate of 0.8 mL/min to equilibrate the medium on a
glass chromatography column. The sample was then loaded and eluted with
Buffer5, while the fraction collector was opened for peak collection (1
mL/tube). Record OD280 - time curve. When the OD280 returns to baseline
and elutes beyond one column volume, stop the elution. Finally determine
the fraction absorption peak corresponding to the lysozyme.

\subsubsection{Detection of molecular sieve effects by SDS-PAGE}

We first observed the results of molecular sieve purification by
SDS-polyacrylamide gel electrophoresis analysis, after which we selected
a single fraction of the electrophoretic band and mixed it well as
Sample3 (measured volume).

\hypertarget{concentration-measurement}{%
	\subsection{Concentration Measurement}\label{concentration-measurement}}

We chose to use the Comas Brilliant Blue G250 staining method to
determine the protein concentration of each sample.

\hypertarget{antibacterial-test}{%
	\subsection{Antibacterial Test}\label{antibacterial-test}}

\subsubsection{Preparation of standard solution for wall-dissolving
	micrococci }

Dissolve the purchased micrococcal powder with Buffer5, grind it
thoroughly in a sterilized mortar, decant and dilute the substrate
suspension to the desired absorbance value.

\subsubsection{Determination of lysozyme activity}

Add 2.5 mL of substrate suspension and 0.5 mL of standard enzyme
solution to a 1 cm cell at 25℃, mix well and count the blanks, measure
the absorbance every 15 s for a total of 6 times. pH=6.4).

\begin{longtable}[]{@{}llll@{}}
	\toprule
	Sample & Sample1 & Sample2 & Sample3\tabularnewline
	\midrule
	\endhead
	Micrococcus lysozyme & 800 & 800 & 800\tabularnewline
	Buffer5 & 195 & 195 & 185\tabularnewline
	Sample volume & 5 & 5 & 5\tabularnewline
	\bottomrule
\end{longtable}

\subsubsection{Definition of enzyme activity unit.}

At room temperature 25°C and pH 6.2, at a wavelength of 450 nm, a
decrease of 0.001 in absorption caused by each min is one enzyme
activity unit, then the activity unit U per mg of lysozyme is


\section{Summary}

